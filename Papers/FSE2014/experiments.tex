
\section{Experiment Setup}
We apply our approach on four subject applications: \emph{espresso}, \emph{space}, \emph{cfrac} and \emph{gawk}.
\begin{table}[htbp]
\centering
\caption{subject applications}
\label{tab_sub_app}
\resizebox{0.5\textwidth}{!}{
\begin{tabular}{c|c|c|c}
\hline
Name & Loc & Number of testcases & Type \\
\hline
espresso & 13256 & 19 & Digital electronic gate circuits simplification\\
\hline
cfrac & 6040 & 2 & Big integer factorization\\
\hline
space & 5846 & 3 & Astronautics interpretation\\
\hline
gawk & 45241 & 20 & String processing\\
\hline
\end{tabular}}
\end{table}
\emph{Espresso} is a fast application for simplying complex digital electronic gate circuits and \emph{cfrac} is a factorization application for big integers. We obtain these two applications from the benchmarks of the \emph{DieHard} project, which contains the testcases we used in this work. \emph{Space} is a well known real world application in astronautics. We obtain this program from the SIR repository, where is also the origin of the testcases for it in this paper. \emph{Gawk} is the GNU \emph{awk} implementation for strings processing. We collect this application from the GNU archives, and generate some testcases for it on our own.

The experiments are run on a 64-bit Ubuntu 13.10 system with 7.7 GiB memory and 4 Intel i7-3770 CPU cores. The compiler gcc 4.8.1 is used and \emph{dlmalloc} version 2.8.6 is our base allocator, which is compiled with -O3 option.

