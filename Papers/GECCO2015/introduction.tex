
\section{Introduction}

To achieve optimal performance, many software systems need to be configured according to workload and running environment. 
Software developers often expose a set of parameters for users to re-configure such software systems adaptively.
However, manual parameter tuning is a demanding challenge because users are usually required, not only to have extensive knowledge about the system and the workload, but also to balance many competing objectives, such as memory consumption and execution time.

Many studies have reported on the challenge of automated parameter tuning \cite{Hoffmann:2011:DKR:1961296.1950390, Vuduc01statisticalmodels,autotuning,Whaley:1998:ATL:509058.509096,Tapus:2002:AHT:762761.762771, hutter2009paramils,arcuri-ssbse-2011,Hoffmann:2011:DKR:1961296.1950390}. Early work attempted to find optimal values with mathmatical approaches \cite{Vuduc01statisticalmodels,autotuning,Whaley:1998:ATL:509058.509096,Tapus:2002:AHT:762761.762771} while SBSE \cite{Harman:2007:CSF:1253532.1254729} has been used in more recent research \cite{hutter2009paramils,arcuri-ssbse-2011,Hoffmann:2011:DKR:1961296.1950390} of this problem. Although these approaches can automatically re-configure a system, their improvements are limitted to those achievable with the given parameters.


Many software systems contain undocumented internal variables and expressions that could also affect the performance of the systems. Thus, these elements could also be good candidates for automated parameter tuning. However, many of these elements are `private', which cannot be changed by changing the exposed parameters or API calls. Moreover, identifying these variables and expressions is very difficult for general users, as it requires a deep understanding of the source code of the system. 


In this paper, we propose an automatic technique to discover internal variables and expressions that normally cannot be accessed directly, but impact on non-functional properties of interest. Our goal is to expose new parameters which can directly influence the values of these internal variables and expressions. To distinguish from parameters exposed by software designer (which we call ``Shallow Parameters''), we say these post-exposed parameters ``Deep Parameters''. Modifying shallow parameter values cannot directly change the internal code elements represented by deep parameters. therefore deep parameters provide additional opportunities for subsequent automated parameter tuning.


There has been an attempt to automate the process of exposing deep parameters with the Software Tuning Panel for Autonomic Control (STAC) \cite{Brake:2008:ADS:1370018.1370031}. STAC first generates a design graph for a subject under optimisation. The design graph represents data reference transition flows in the subject. It then uses the reference patterns of shallow parameters to discover deep parameters, whose reference pattern is the same as one of the shallow parameter reference patterns. Although STAC can discover some deep parameters effectively, it suffers from two limitations. First, STAC requires initial human effort to characterise shallow parameters. Second, STAC can only find a subset of deep parameters, those that have similar data transition patterns to the known shallow parameters. To overcome these limitations, we apply a mutation-based sensitivity analysis to fully automated the process of locating potential deep parameters and subsequently apply NSGA II to search for optimal values for these parameters to balance non-functional properties of interest. 

In this paper, we focused on two non-functional properties, memory consumption and execution time, because they are important objectives for many applications and because they are naturally conflicting. We illustrate the approach by re-configuring a general purpose memory allocator, \emph{dlmalloc}. We choose memory allocators as a case study, because memory allocators are critical to the memory consumption of a program and could take up to 60\% of the total execution time \cite{Zorn:1992:EMS:142181.142200}. As a result, memory optimisation is a widely studied topic \cite{Risco-Martin:2009:ODM:1569901.1570116,RiscoMartin2010572}. We evaluate our approach using four applications including benchmarks for \emph{dlmalloc} and real world applications.

The paper presents evidence to suggest that deep parameter optimisation is an effective approach to improve non-function properties for \emph{dlmalloc}. 
We report experiments that show deep parameter optimisation competes with shallow parameter optimisation and the default optimised configurations. For all of our subjects, the deep parameter optimisation saves x\% of memory and reduces y\% of execution time in the best case. We also investigate the deep parameters exposed by our approach are meaningful to human developers. The contributions of the paper are summarised as follow.


\begin{enumerate}

\item We introduce an automated approach to discover deep parameters. 

\item We report the results of an empirical study comparing the shallow parameter tuning approach with our approach. On 4 real world applications, the results show that our approach can save x\% of memory and reduce y\% of execution time, whereas shallow tuning alone achieves p\% time and q\% space reduction. 

\item We also report a further case study on the deep parameters found by the mutation based sensitivity analysis. The results suggest that these parameters are meaningfull to human developers and are good candidates to be promoted into explicit parameters. 

\end{enumerate}

%The rest of this paper is organised as follows. 
%Section 2 introduces the background of memory management and a popular general-purpose memory allocator \emph{dlmalloc}.
%Section 3 describes out adaptive deep parameter tuning approach.
%Section 4 explains the experimental methods for the empirical study, the results of which are discussed in Section 5
%Section 6 discusses related work, and the paper concludes with Section 7.


