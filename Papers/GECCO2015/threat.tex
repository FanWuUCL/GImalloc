\subsection{Threats to Validity}

\textbf{Internal Validity}  When exposing deep parameters, we used Mutation-based sensitivity analysis because it advantages in efficiency and automation. but whether it is the best way to expose deep parameters needs to be proven, as well as which Mutation Operators shall be used. Intuitively, Mutation Operators change a constant or an operator in an arithmatic expression, thus likely to change the value of the expression in different degrees, which allow us to capture the sensitivity of this expression. How good Mutation Operators are in capturing sensitivity information remains as future work.
Another threat to the internal validity is that the execution time measured may depend on the workload of the machine, thus it may slightly vary over time due to other works of the Operating System. By averaging 10 times of measurement, the noise were minimized in our experiments.  

\textbf{External Validity}  The choose of test suite could in some degree affect the results. Even with a good test suite that achieves a high branch coverage, it could still differ from the distribution of the real world inputs, in which case the optimized configuration over this testsuite may not achieve the best performance in real world situations. 
One other concern is whether the result holds on other applications. Despite subject applications from different fields for different uses, potential threat to our conclusion could exist beyond our test. Currently we can only claim that our approach works on the applications under test in this paper, but because of the wide range of where these applications come from, the approach is likely to be generalized on other applications.
