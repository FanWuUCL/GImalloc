
\section{Introduction}

To achieve optimal performance, many software systems need to be configured according to various workload and running environment. 
Software developers often expose a set of parameters for users to re-configure such software systems adaptively.
However, manual parameter tuning is a demanding challenge. Because users are required to not only have extensive knowledge about the system and the workload but also to balance many competing objectives, such as memory consumption and execution time.

Many work has risen to the challenge of automated parameter tuning\cite{Hoffmann:2011:DKR:1961296.1950390}. Early work attempted to find optimal values with mathmatical approaches \cite{Vuduc01statisticalmodels,autotuning,Whaley:1998:ATL:509058.509096,Tapus:2002:AHT:762761.762771} while SBSE\cite{Harman:2007:CSF:1253532.1254729} has been used in more recent research \cite{hutter2009paramils,arcuri-ssbse-2011,Hoffmann:2011:DKR:1961296.1950390}. Although these approaches can automatically re-configure a system under optimisation, they only provide limited improvements because they can only turn the parameters specified by system designer.

Many software systems contain undocumented internal variables and expressions which could also affect the performance of the software. Thus, these elements could also be good candidates for automated parameter tuning. However, many of these elements are `private', which cannot be changed from changing the exposed parameters or API calls. Moreover, identifying these variables and expressions is very difficult for general users, as it requires a deep understanding of the source code of the system. 

In this paper, we propose an automatic technique to discover the internal variables and expressions which cannot be accessed from changing the exposed parameters, but have greatly impacts on non-functional properties of interests. We then expose new parameters which can be used to change the values for these hidden variables and expressions. To distinguish with parameters exposed by software designer (Shallow Parameters), we name these new parameters Deep Parameters. As changing shallow parameters cannot directly change the internal code elements represented by deep parameters, the deep parameters thus provide additional opportunities in automated parameter tuning.

There has been an attempt to automate the process of exposing deep parameters with Software Tuning Panel for Autonomic Control (STAC)\cite{Brake:2008:ADS:1370018.1370031}. STAC first generates a design graph for a subject under optimisation. The design graph represents data reference transition flows in the subject. It then converts shallow parameters into sub-graph patterns. To discover deep parameters, STAC searches for similar sub-graph patterns in the complete design graph. 

Although STAC can discover some deep parameters effectively, it surfers from two limitations. First, STAC requires initial human effort to characterise shallow parameters. Second, STAC can only find a subset of deep parameters which have similar data transition patterns as the known shallow parameters. To overcome these limitations, we apply a mutation based sensitivity analysis to fully automated the process of locating the potential deep parameters and applies NSGAII to search for  optimal values to balance non-functional properties of interests. 

We illustrate the approach by reconfiguring a general purpose memory allocator,\emph{Dlmalloc}. We choose this application as an example, because many research has attempted to turn \emph{Dlmalloc} \cite{Risco-Martin:2009:ODM:1569901.1570116,RiscoMartin2010572}. In this paper, we focused on two non-functional properties, memory consumption and execution time, which are essential objectives to memory allocator. We evaluate our approach using four applications including benchmarks for \emph{Dlmalloc} and real world applications.

The paper presents evidence to suggest that the deep parameter optimisation is an effective approach to improve non-function properties for \emph{dlmalloc}. 
We report experiments that show the deep parameter optimisation competes with the shallow parameter optimisation and the default configurations. For all of our subjects, the deep parameter optimisation saves x\% of memory and reduces y\% of execution time overall. Moreover we also present reasons why the implicit parameters exposed by our approach are meaningful to human developer. The contributions of the paper be summarised as follow.


\begin{enumerate}

\item We introduce a automated approach to discover deep paramerts. 

\item We report the results of an empirical study comparing the shallow parameter tuning approach with our approach. On \# real world applications, the results show that our approach can save x\% of memory and reduce y\% of execution time over shallow parameter tuning. 

\item We also report a further case study on the implicity parameters found by the mutation based sensitivity analysis. The results suggest that these parameter are meaningfull to human developer and are good candidates to be promoted into explicit parameters. 

\end{enumerate}

%The rest of this paper is organised as follows. 
%Section 2 introduces the background of memory management and a popular general-purpose memory allocator \emph{dlmalloc}.
%Section 3 describes out adaptive deep parameter tuning approach.
%Section 4 explains the experimental methods for the empirical study, the results of which are discussed in Section 5
%Section 6 discusses related work, and the paper concludes with Section 7.


