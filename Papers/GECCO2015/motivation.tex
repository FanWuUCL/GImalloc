\vspace{-2.0em}
\section{Motivating Example}

We illustrate the idea of deep parameters with an example found by our approach for \emph{dlmalloc} (version 2.8.6)~\cite{lea1996memory}.


\begin{figure}[ht]
\vspace{-1.5em}
\lstset{numbers=left, escapeinside={(*@}{@*)}}
\begin{lstlisting}
static void* sys_alloc(mstate m,size_t nb) {
...
	if (ss == 0){ //check if first time through
		char* base = (char*)CALL_MORECORE(0); (*@\label{call_morecore}@*)
	...
}
\end{lstlisting}
\vspace{-2em}
\caption{{\tt sys\_alloc} function in \emph{dlmalloc}}
\label{exp}
\end{figure}
\vspace{-1em}

Figure~\ref{exp} shows a part of the {\tt sys\_alloc} function in \emph{dlmalloc}. We explain its internal operation here to give the reader a feeling for the opportunities for optimisation. Of course, our parameter exposing and search-based tuning are general purpose techniques that have no knowledge of how \emph{dlmalloc} operates. \emph{Dlmalloc} maintains an internal structure to organize the heap for memory reuse. Only when \emph{dlmalloc} cannot find a suitable chunk of memory for a memory request does it call {\tt sys\_alloc()} to extend the current heap.

% Fan, the "mutation analysis" bit in the next paragraph is quite
% "Used before being defined". I added this paragraph to make things make
% more sense, but please correct it -- I may not be describing your process
% accurately. 

% Fan checked the paragraph and it is accurate.

Our approach begins with a form of mutation analysis that evaluates
subexpressions in the program to determine their utility as candidate
deep parameters. A subexpression is evaluated by mutating it, running
the resulting program variant against a test suite, and evaluating the
results in terms of functional and non-functional properties. A
subexpression that can be profitably mutated to optimise a non-functional
property while retaining functional correctness can serve as a deep
parameter.  

In this example, the mutation analysis finds that mutants generated from mutating Line~\ref{call_morecore} have a notable affect on the memory consumption and the execution time of \emph{dlmalloc}. We take a close took at Line~\ref{call_morecore}. It calls the {\tt CALL\_MORECORE()} function, which takes an integer as input.
% , when first time \emph{malloc} is called to obtain the beginning address of the heap. 
{\tt CALL\_MORECORE()} is a macro wrapping the system call that extends or shrinks the current heap and returns the beginning address of the newly allocated region of heap. Specifically, {\tt CALL\_MORECORE(0)} neither extends nor shrinks the heap but simply returns the current address of the heap, which is the original purpose of Line~\ref{call_morecore} mentioned above.

Changing the input value for {\tt CALL\_MORECORE()} in Line~\ref{call_morecore} 
allows us to control the amount of memory
pre-allocated. However, although \emph{dlmalloc} provides several tuneable
parameters to programmers, allowing them to adjust behaviours (see Section 5
for details), none of these shallow parameters can affect the {\tt
CALL\_MORECORE()} function directly. Our algorithm exposes this as a new
deep parameter by 
transforming Line~\ref{call_morecore} into
the code below, where $D$ is the deep parameter exposed that controls the
pre-allocated heap.
\vspace{-0.4em}
\begin{lstlisting}
char * base = (char*)CALL_MORECORE(0 + D);
\end{lstlisting}
\vspace{-0.4em}

The optimal size of pre-allocated memory depends on the specific program using this tunable memory allocator. Too much pre-allocation may result in waste. On the other hand, too little means that later requests must call {\tt CALL\_MORECORE()} again to extend the heap, increasing runtime. By tuning the deep parameter $D$, an SBSE approach can balance time and space consumption. This is just one example of a potential deep parameter. In our mutation analysis experiments, our tool `discovers' that by changing the value of this deep parameter, it can achieve a modest (2.5\%) time reduction without increasing heap space in one of our subjects.

