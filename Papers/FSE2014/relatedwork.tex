\section{Related Work}

Some embedded systems, especially those executing multimedia applications, suffer from massive memory usage and limited resources. Risco-Martin et al\cite{Risco-Martin:2009:ODM:1569901.1570116}\cite{RiscoMartin2010572} decomposes memory allocators into several components, for each of which there are several optional implementations of different allocation strategies. Combining different implementations to generate the optimal dynamic memory manager (DMM) becomes a searching problem. They use grammatical evolution to solve this optimization problem with two real world applications: Physics3D and VDrift. Other than \emph{dlmalloc}, they target the DMM on embedded systems that usually run memory-intensive applications. In their approach, they try to find the best combination of several basic strategies, different from which, we start from the state-of-the-art combination of allocation strategies and adjust its configuration to each application. 

Grunwald and Zorn introduced \emph{CustoMalloc}, a system that customizes and synthesizes a memory allocator for a given application\cite{SPE:SPE4380230804}. The basic idea is, run an application and record all the memory allocation and deallocation during the run so that \emph{CustoMalloc} can find the most frequent sizes. Then the system generates a custom memory allocator using two allocation strategies for different sizes: fast but more overheaded way for the most frequent sizes and traditional way for other sizes. They also reported that the performance of a synthesized allocator is not sensitive to the input of the application, suggesting that for a given application, the memory allocation and deallocation patterns for different inputs are similar. In their and our works, we both try to create a custom memory allocator for each specific application but we use a simpler way, parameter tuning in one of the best allocators. Extending from that, we expose more valuable parameters from the source code and optimize them.

Speaking of parameter tuning, there has been many works studying the influence of algorithms' configuration or automatically adjusting it, including \emph{ParamILS}\cite{hutter2009paramils}. \emph{ParamILS} is an automatic framework proposed by Hutter et al, which automatically configures an algorithm's parameters to get the best performance on a given test suite. It uses a local-search-based algorithm to look for the optimum and gets the fitness by running the application with each candidate configuration. Despite as well parameter tuning, we instead focus on the standard library code, considering the general-purpose memory allocator may not be the optimal for each specific application. What's more, \emph{ParamILS} can only optimize existing parameters. 

Hoffmann and Sidiroglou et al\cite{Hoffmann:2011:DKR:1961296.1950390} proposed \emph{PowerDial}, a system which dynamically adjusts application's behavior to make it adaptable to fluctuating working load and power. It first transforms some configuration parameters to non-constant variables residing in the application's memory, so the behavior of the application can be altered by controling these variables at runtime. Then it pre-runs the application with each possible configuration to abtain how these parameters influence the application, memorizes the Pareto-best candidates in terms of application's non-functional properties and the quality of the output. Whenever \emph{PowerDial} detects a resource shortage it sacrifices some of the output quality by changing the values of those variables according to its record, to make the application survive from the crisis. One limitation of this work is that, the options of these variables can not be too large to thoroughly try each of them. In our work the search space is so massive that it could take dozens of years to exhaust the whole space. \emph{PowerDial} as well only works on the existing parameters.
